% !TeX spellcheck = en_US
%% 字体:方正静蕾简体
%%		 方正粗宋
\documentclass[a4paper,left=2.5cm,right=2.5cm,11pt]{article}

\usepackage[utf8]{inputenc}
\usepackage{fontspec}
\usepackage{cite}
\usepackage{xeCJK}
\usepackage{indentfirst}
\usepackage{titlesec}
\usepackage{longtable}
\usepackage{graphicx}
\usepackage{float}
\usepackage{rotating}
\usepackage{subfigure}
\usepackage{tabu}
\usepackage{amsmath}
\usepackage{setspace}
\usepackage{amsfonts}
\usepackage{appendix}
\usepackage{listings}
\usepackage{xcolor}
\usepackage{geometry}
\setcounter{secnumdepth}{4}
\usepackage{mhchem}
\usepackage{multirow}
\usepackage{extarrows}
\usepackage{hyperref}
\titleformat*{\section}{\LARGE}
\renewcommand\refname{参考文献}
\renewcommand{\abstractname}{\sihao \cjkfzcs 摘{  }要}
%\titleformat{\chapter}{\centering\bfseries\huge\wryh}{}{0.7em}{}{}
%\titleformat{\section}{\LARGE\bf}{\thesection}{1em}{}{}
\titleformat{\subsection}{\Large\bfseries}{\thesubsection}{1em}{}{}
\titleformat{\subsubsection}{\large\bfseries}{\thesubsubsection}{1em}{}{}
\renewcommand{\contentsname}{{\cjkfzcs \centerline{目{  } 录}}}
\setCJKfamilyfont{cjkhwxk}{STXingkai}
\setCJKfamilyfont{cjkfzcs}{STSongti-SC-Regular}
% \setCJKfamilyfont{cjkhwxk}{华文行楷}
% \setCJKfamilyfont{cjkfzcs}{方正粗宋简体}
\newcommand*{\cjkfzcs}{\CJKfamily{cjkfzcs}}
\newcommand*{\cjkhwxk}{\CJKfamily{cjkhwxk}}
\newfontfamily\wryh{Microsoft YaHei}
\newfontfamily\hwzs{STZhongsong}
\newfontfamily\hwst{STSong}
\newfontfamily\hwfs{STFangsong}
\newfontfamily\jljt{MicrosoftYaHei}
\newfontfamily\hwxk{STXingkai}
% \newfontfamily\hwzs{华文中宋}
% \newfontfamily\hwst{华文宋体}
% \newfontfamily\hwfs{华文仿宋}
% \newfontfamily\jljt{方正静蕾简体}
% \newfontfamily\hwxk{华文行楷}
\newcommand{\verylarge}{\fontsize{60pt}{\baselineskip}\selectfont}  
\newcommand{\chuhao}{\fontsize{44.9pt}{\baselineskip}\selectfont}  
\newcommand{\xiaochu}{\fontsize{38.5pt}{\baselineskip}\selectfont}  
\newcommand{\yihao}{\fontsize{27.8pt}{\baselineskip}\selectfont}  
\newcommand{\xiaoyi}{\fontsize{25.7pt}{\baselineskip}\selectfont}  
\newcommand{\erhao}{\fontsize{23.5pt}{\baselineskip}\selectfont}  
\newcommand{\xiaoerhao}{\fontsize{19.3pt}{\baselineskip}\selectfont} 
\newcommand{\sihao}{\fontsize{14pt}{\baselineskip}\selectfont}      % 字号设置  
\newcommand{\xiaosihao}{\fontsize{12pt}{\baselineskip}\selectfont}  % 字号设置  
\newcommand{\wuhao}{\fontsize{10.5pt}{\baselineskip}\selectfont}    % 字号设置  
\newcommand{\xiaowuhao}{\fontsize{9pt}{\baselineskip}\selectfont}   % 字号设置  
\newcommand{\liuhao}{\fontsize{7.875pt}{\baselineskip}\selectfont}  % 字号设置  
\newcommand{\qihao}{\fontsize{5.25pt}{\baselineskip}\selectfont}    % 字号设置 

\usepackage{diagbox}
\usepackage{multirow}
\boldmath
\XeTeXlinebreaklocale "zh"
\XeTeXlinebreakskip = 0pt plus 1pt minus 0.1pt
\definecolor{cred}{rgb}{0.8,0.8,0.8}
\definecolor{cgreen}{rgb}{0,0.3,0}
\definecolor{cpurple}{rgb}{0.5,0,0.35}
\definecolor{cdocblue}{rgb}{0,0,0.3}
\definecolor{cdark}{rgb}{0.95,1.0,1.0}
\lstset{
	language=bash,
	numbers=left,
	numberstyle=\tiny\color{black},
	showspaces=false,
	showstringspaces=false,
	basicstyle=\scriptsize,
	keywordstyle=\color{purple},
	commentstyle=\itshape\color{cgreen},
	stringstyle=\color{blue},
	frame=lines,
	% escapeinside=``,
	extendedchars=true, 
	xleftmargin=1em,
	xrightmargin=1em, 
	backgroundcolor=\color{cred},
	aboveskip=1em,
	breaklines=true,
	tabsize=4
} 

\newfontfamily{\consolas}{Consolas}
\newfontfamily{\monaco}{Monaco}
\setmonofont[Mapping={}]{Consolas}	%英文引号之类的正常显示,相当于设置英文字体
\setsansfont{Consolas} %设置英文字体 Monaco, Consolas,  Fantasque Sans Mono
\setmainfont{Times New Roman}

\setCJKmainfont{华文中宋}


\newcommand{\fic}[1]{\begin{figure}[H]
		\center
		\includegraphics[width=0.8\textwidth]{#1}
	\end{figure}}
	
\newcommand{\sizedfic}[2]{\begin{figure}[H]
		\center
		\includegraphics[width=#1\textwidth]{#2}
	\end{figure}}

\newcommand{\codefile}[1]{\lstinputlisting{#1}}

\newcommand{\interval}{\vspace{0.5em}}

\newcommand{\tablestart}{
	\interval
	\begin{longtable}{p{2cm}p{10cm}}
	\hline}
\newcommand{\tableend}{
	\hline
	\end{longtable}
	\interval}

% 改变段间隔
\setlength{\parskip}{0.2em}
\linespread{1.1}

\usepackage{lastpage}
\usepackage{fancyhdr}
\pagestyle{fancy}
\lhead{\space \qquad \space}
\chead{linux的环境变量 \qquad}
\rhead{\qquad\thepage/\pageref{LastPage}}
\begin{document}

\tableofcontents

\clearpage

\section{环境变量的介绍}
	linux中有两种环境变量,分别是全局变量和本地变量。

\subsection{全局环境变量}
	linux系统本身就有初始化一些全局环境变量,这些系统环境变量都是全大写字母。可以通过printenv命令查看全局环境变量,如下图所示:
	\sizedfic{1}{1.png}

	可以通过在变量前加\$来查看变量的值,例子如下:
	\begin{lstlisting}
	echo $HOME
	\end{lstlisting}

\subsection{本地环境变量}
	本地环境变量只在它们的本地进程中可见,通过set命令可以查看所有本地定义的shell变量。

\section{设置环境变量}
\subsection{设置本地环境变量}
	本地环境变量一般使用小写,定义方法如下所示:
	\begin{lstlisting}
	test='testing a long string'
	\end{lstlisting}

	需要注意的是,环境变量名称、等号和值之间是没有空格的。\par

	通过\$可以查看变量的值,如下所示:
	\begin{lstlisting}
	echo $test
	\end{lstlisting}

	本地环境变量只在当前shell进程中可见,其他shell进程中没有这个变量。

\subsection{设置全局环境变量}
	创建全局环境变量的方法是创建一个本地环境变量,然后再将它导出到全局环境中。方法如下所示:
	\begin{lstlisting}
	test='testing a long string'
	export test
	\end{lstlisting}

	此时,如果再开启另一个shell进程,也可以看到test环境变量了,如下所示:
	\begin{lstlisting}
	bash
	echo $test
	\end{lstlisting}

	需要注意的是,这里的全局环境变量只是暂时的,进程退出后就会被移除。

\section{移除环境变量}
	通过unset命令可以移除环境变量,如下所示:
	\begin{lstlisting}
	# 假设之前定义过一个test变量
	unset test
	\end{lstlisting}

	需要注意的是,unset只能移除本地环境变量,不能移除全局环境变量。
	在子进程移除全局环境变量后,只能保证子进程中没有这个变量,而父进程仍然存在这个环境变量。

\section{默认的shell变量}
	系统环境变量如下所示:
	\begin{longtable}{p{3cm}p{9cm}}
	\hline
	CDPATH & 冒号分隔的目录列表,用作cd命令的搜索路径 \\
	\hline
	HOME & 当前用户的主目录 \\
	\hline
	IFS & 用于分隔字段的字符列表 \\
	\hline
	MAIL & 当前用户邮箱的文件名 \\
	\hline
	MAILPATH & 当前用户邮箱的多个文件名 \\
	\hline
	OPTARG & getopts命令处理的最后一个选项参数的值 \\
	\hline
	OPTIND & getopts命令处理的最后一个选项参数的索引值 \\
	\hline
	PATH & 冒号分隔的目录列表,shell将在这些目录中查找命令 \\
	\hline
	PS1 & 主shell命令行界面提示字符串 \\
	\hline
	PS2 & 次shell命令行界面提示字符串 \\
	\hline
	BASH & 执行当前bash shell的完整路径名称 \\
	\hline
	BASH\_ENV & 各bash脚本将在运行之前尝试执行此变量定义的启动文件 \\
	\hline
	BASH\_VERSION & 当前bash shell的版本 \\
	\hline
	BASH\_VERSINFO & 包含当前bash shell的主要及次要版本的变量数组 \\
	\hline
	COLUMNS & 当前bash shell所使用的终端的终端宽度 \\
	\hline
	COMP\_CWORD & 当前光标的位置 \\
	\hline
	COMP\_LINE & 当前命令行 \\
	\hline
	COMP\_POINT & 当前光标相对于当前命令开始处的位置的索引 \\
	\hline
	COMP\_WORDS & 当前命令行上的各单词的变量数组 \\
	\hline
	COMPREPLY & 由shell函数生成的完成代码的变量数组 \\
	\hline
	DIRSTACK & 包含目录栈当前内容的变量数组 \\
	\hline
	EUID & 当前用户数值形式的有效用户ID \\
	\hline
	FCEDIT & fc命令使用的默认编辑器 \\
	\hline
	FIGNORE & 冒号分隔的后缀列表,将在执行文件名完成时忽略 \\
	\hline
	FUNCNAME & 当前正在执行的shell函数的名称 \\
	\hline
	GLOBIGNORE & 冒号分隔的后缀列表,用于定义文件扩展忽略的文件名集合 \\
	\hline
	GROUPS & 包含当前用户所在用户组的变量数组 \\
	\hline
	Histchars & 最多3个字符,用于控制历史扩展 \\
	\hline
	HISTCMD & 当前命令的历史记录数量 \\
	\hline
	HISTCONTROL & 控制shell历史列表中输入的命令 \\
	\hline
	HISTFILE & 保存shell历史记录的文件的名称 \\
	\hline
	HISTFILESIZE & 历史文件中可保存的最大行数 \\
	\hline
	HISTIGNORE & 冒号分隔的模式列表,用于定义哪些命令将被历史文件忽略 \\
	\hline
	HISTSIZE & 存储在历史文件中的最大命令数量 \\
	\hline
	HOSTFILE & 当shell需要完成主机名时应该读取的文件的名称 \\
	\hline
	HOSTNAME & 当前主机的名称 \\
	\hline
	HOSTTYPE & 运行bash shell的机器的字符串 \\
	\hline
	IGNOREEOF & shell在退出之前必须接受的连续EOF字符的数量 \\
	\hline
	INPUTRC & Readline初始化文件的名称 \\
	\hline
	LANG & shell的地区类别 \\
	\hline
	LC\_ALL & 覆盖LANG变量,定义地区类别 \\
	\hline
	LC\_COLLATE & 设置对字符串值进行排序时所使用的排序顺序 \\
	\hline
	LC\_CTYPE & 确定在文件名扩展和模式匹配中使用的字符译码 \\
	\hline
	LC\_MESSAGES & 确定在解释双引号字符串前出现美元符号时使用的地区设置 \\
	\hline
	LC\_NUMERIC & 确定格式化数字时使用的地区设置 \\
	\hline
	LINENO & 当前执行脚本中的行号 \\
	\hline
	LINES & 定义终端上可用的行数 \\
	\hline
	MACHTYPE & 以cpu-company-system格式定义系统类型的字符串 \\
	\hline
	MAILCHECK & shell应该检查新邮件的频率 \\
	\hline
	OLDPWD & shell中刚使用过的工作目录 \\
	\hline
	OPTERR & 如果设置为1,则bash shell将显示getopts命令生成的错误 \\
	\hline
	OSTYPE & 运行shell的操作系统 \\
	\hline
	PIPESTATUS & 包含前台进程中的进程退出状态值列表的变量数组 \\
	\hline
	POSIXLY\_CORRECT & 如果设置了这个变量,则bash将以POSIX模式启动 \\
	\hline
	PPID & bash父进程的进程ID \\
	\hline
	PROMPT\_COMMAND & 在显示主提示符之前执行的命令 \\
	\hline
	PS3 & select命令使用的提示符 \\
	\hline
	PS4 & 回显命令行之前显示的提示符 \\
	\hline
	PWD & 当前工作目录 \\
	\hline
	RANDOM & 返回0到32767之间的随机数 \\
	\hline
	REPLY & read命令的默认变量 \\
	\hline
	SECONDS & 自shell启动之后的秒数 \\
	\hline
	SHELLOPTS & 已启动的bash shell选项列表 \\
	\hline
	SHLVL & 指示shell级别,每次开始一个新bash shell时,该变量加1 \\
	\hline
	TIMEFORMAT & 指定shell如何显示时间值的格式 \\
	\hline
	TMOUT & select和read命令应该等待输入的时间 \\
	\hline
	UID & 当前用户数值形式的真实用户id \\
	\hline
	\end{longtable}

\section{设置PATH变量}
	PATH定义了shell在哪些目录下查找在命令行中输入的命令。我们可以人为向PATH中添加搜索目录,命令如下:
	\begin{lstlisting}
	# /home/pengsida/kvm是我们想添加的搜索目录
	PATH=$PATH:/home/pengsida/kvm
	\end{lstlisting}

	单个点符号代表当前目录,可以在PATH环境变量中包含一个点符号,命令如下:
	\begin{lstlisting}
	PATH=$PATH:.
	\end{lstlisting}

\section{更改系统环境变量}
	根据bash shell的不同,相应的启动文件也不同。系统中有三种bash shell,如下所示:
	\begin{itemize}
		\item[1.] 登陆系统后的交互式shell。
		\item[2.] 未登陆系统时的交互式shell。
		\item[3.] 非交互式shell。
	\end{itemize}

\subsection{登陆系统后的交互式shell}
	此时bash shell有4个启动文件,bash shell将依次处理以下4个启动文件:
	\begin{itemize}
		\item[1.] /etc/profile
		\item[2.] \$HOME/.bash\_profile
		\item[3.] \$HOME/.bash\_login
		\item[4.] \$HOME/.profile
	\end{itemize}

\subsubsection{/etc/profile文件}
	/etc/profile文件是bash shell的主默认启动文件,查看它的内容的命令如下:
	\begin{lstlisting}
	cat /etc/profile
	\end{lstlisting}

	可以在该文件中添加如下代码:
	\begin{lstlisting}
	# VAR1、VAR2、VAR3是想添加的全局变量
	export VAR1="VAR1" VAR2="VAR2" VAR3="VAR3"
	\end{lstlisting}

	这样的话,就可以永久地将VAR1、VAR2、VAR3设置为全局变量,这些环境变量对于shell生成的子进程都可用。

\subsubsection{\$HOME启动文件}
	\$HOME启动文件有\$HOME/.bash\_profile、\$HOME/.bash\_login和\$HOME/.profile。和/etc/profile一样,也可以在这些文件中添加全局变量,只要添加如下代码:
	\begin{lstlisting}
	PATH=$PATH:$HOME/bin
	export PATH
	\end{lstlisting}

\subsection{未登陆时的交互式shell}
	当启动了一个bash shell而没有登陆系统时,这个bash shell需要检查用户HOME目录中的.bashrc文件,可以通过如下命令查看这个文件的内容:
	\begin{lstlisting}
	cat $HOME/.bashrc
	\end{lstlisting}

	这个文件首先将检查/etc/bash.bashrc文件,这个文件可以用于添加全局变量。我们还可以在.bashrc文件中设置个人别名和私有脚本函数。

\subsection{非交互式shell}
	我们无法对这个shell输入命令,是非交互式的。这个bash shell将执行由BASH\_ENV变量指定的启动文件。

\section{变量数组}
	变量数组的形式如下所示:
	\begin{lstlisting}
	# 值在圆括号中,各值以空格分隔
	mytest=(one two three four five)
	\end{lstlisting}

	索引变量数组的值的方式如下所示:
	\begin{lstlisting}
	# 2是索引值,输出结果是three
	echo $(mytest[2])
	\end{lstlisting}

	如果要显示整个数组的变量,可以使用星号通配符作为索引值,如下所示:
	\begin{lstlisting}
	echo $(mytest[*])
	\end{lstlisting}

	可以通过unset命令将数组中某位置上的值设为NULL,如下所示:
	\begin{lstlisting}
	unset mytest[2]
	\end{lstlisting}

	还可以通过unset命令将数组中的所有值都设为NULL,如下所示:
	\begin{lstlisting}
	unset mytest
	\end{lstlisting}

\section{使用命令别名}
	命令别名指的是命令的别名,linux中本身就有一些公共命令别名,可以通过如下命令查看:
	\begin{lstlisting}
	alias -p
	\end{lstlisting}

	可以通过alias命令来创建自己的别名,如下所示:
	\begin{lstlisting}
	# li就是我们定义的别名
	alias li='ls -il'
	\end{lstlisting}

	需要注意的是,命令别名与本地环境变量类似,只对定义范围内的shell进程有效。
  
\end{document}
