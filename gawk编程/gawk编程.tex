% !TeX spellcheck = en_US
%% 字体:方正静蕾简体
%%		 方正粗宋
\documentclass[a4paper,left=2.5cm,right=2.5cm,11pt]{article}

\usepackage[utf8]{inputenc}
\usepackage{fontspec}
\usepackage{cite}
\usepackage{xeCJK}
\usepackage{indentfirst}
\usepackage{titlesec}
\usepackage{longtable}
\usepackage{graphicx}
\usepackage{float}
\usepackage{rotating}
\usepackage{subfigure}
\usepackage{tabu}
\usepackage{amsmath}
\usepackage{setspace}
\usepackage{amsfonts}
\usepackage{appendix}
\usepackage{listings}
\usepackage{xcolor}
\usepackage{geometry}
\setcounter{secnumdepth}{4}
\usepackage{mhchem}
\usepackage{multirow}
\usepackage{extarrows}
\usepackage{hyperref}
\titleformat*{\section}{\LARGE}
\renewcommand\refname{参考文献}
\renewcommand{\abstractname}{\sihao \cjkfzcs 摘{  }要}
%\titleformat{\chapter}{\centering\bfseries\huge\wryh}{}{0.7em}{}{}
%\titleformat{\section}{\LARGE\bf}{\thesection}{1em}{}{}
\titleformat{\subsection}{\Large\bfseries}{\thesubsection}{1em}{}{}
\titleformat{\subsubsection}{\large\bfseries}{\thesubsubsection}{1em}{}{}
\renewcommand{\contentsname}{{\cjkfzcs \centerline{目{  } 录}}}
\setCJKfamilyfont{cjkhwxk}{STXingkai}
\setCJKfamilyfont{cjkfzcs}{STSongti-SC-Regular}
% \setCJKfamilyfont{cjkhwxk}{华文行楷}
% \setCJKfamilyfont{cjkfzcs}{方正粗宋简体}
\newcommand*{\cjkfzcs}{\CJKfamily{cjkfzcs}}
\newcommand*{\cjkhwxk}{\CJKfamily{cjkhwxk}}
\newfontfamily\wryh{Microsoft YaHei}
\newfontfamily\hwzs{STZhongsong}
\newfontfamily\hwst{STSong}
\newfontfamily\hwfs{STFangsong}
\newfontfamily\jljt{MicrosoftYaHei}
\newfontfamily\hwxk{STXingkai}
% \newfontfamily\hwzs{华文中宋}
% \newfontfamily\hwst{华文宋体}
% \newfontfamily\hwfs{华文仿宋}
% \newfontfamily\jljt{方正静蕾简体}
% \newfontfamily\hwxk{华文行楷}
\newcommand{\verylarge}{\fontsize{60pt}{\baselineskip}\selectfont}  
\newcommand{\chuhao}{\fontsize{44.9pt}{\baselineskip}\selectfont}  
\newcommand{\xiaochu}{\fontsize{38.5pt}{\baselineskip}\selectfont}  
\newcommand{\yihao}{\fontsize{27.8pt}{\baselineskip}\selectfont}  
\newcommand{\xiaoyi}{\fontsize{25.7pt}{\baselineskip}\selectfont}  
\newcommand{\erhao}{\fontsize{23.5pt}{\baselineskip}\selectfont}  
\newcommand{\xiaoerhao}{\fontsize{19.3pt}{\baselineskip}\selectfont} 
\newcommand{\sihao}{\fontsize{14pt}{\baselineskip}\selectfont}      % 字号设置  
\newcommand{\xiaosihao}{\fontsize{12pt}{\baselineskip}\selectfont}  % 字号设置  
\newcommand{\wuhao}{\fontsize{10.5pt}{\baselineskip}\selectfont}    % 字号设置  
\newcommand{\xiaowuhao}{\fontsize{9pt}{\baselineskip}\selectfont}   % 字号设置  
\newcommand{\liuhao}{\fontsize{7.875pt}{\baselineskip}\selectfont}  % 字号设置  
\newcommand{\qihao}{\fontsize{5.25pt}{\baselineskip}\selectfont}    % 字号设置 

\usepackage{diagbox}
\usepackage{multirow}
\boldmath
\XeTeXlinebreaklocale "zh"
\XeTeXlinebreakskip = 0pt plus 1pt minus 0.1pt
\definecolor{cred}{rgb}{0.8,0.8,0.8}
\definecolor{cgreen}{rgb}{0,0.3,0}
\definecolor{cpurple}{rgb}{0.5,0,0.35}
\definecolor{cdocblue}{rgb}{0,0,0.3}
\definecolor{cdark}{rgb}{0.95,1.0,1.0}
\lstset{
	language=bash,
	numbers=left,
	numberstyle=\tiny\color{black},
	showspaces=false,
	showstringspaces=false,
	basicstyle=\scriptsize,
	keywordstyle=\color{purple},
	commentstyle=\itshape\color{cgreen},
	stringstyle=\color{blue},
	frame=lines,
	% escapeinside=``,
	extendedchars=true, 
	xleftmargin=1em,
	xrightmargin=1em, 
	backgroundcolor=\color{cred},
	aboveskip=1em,
	breaklines=true,
	tabsize=4
} 

\newfontfamily{\consolas}{Consolas}
\newfontfamily{\monaco}{Monaco}
\setmonofont[Mapping={}]{Consolas}	%英文引号之类的正常显示,相当于设置英文字体
\setsansfont{Consolas} %设置英文字体 Monaco, Consolas,  Fantasque Sans Mono
\setmainfont{Times New Roman}

\setCJKmainfont{华文中宋}


\newcommand{\fic}[1]{\begin{figure}[H]
		\center
		\includegraphics[width=0.8\textwidth]{#1}
	\end{figure}}
	
\newcommand{\sizedfic}[2]{\begin{figure}[H]
		\center
		\includegraphics[width=#1\textwidth]{#2}
	\end{figure}}

\newcommand{\codefile}[1]{\lstinputlisting{#1}}

\newcommand{\interval}{\vspace{0.5em}}

\newcommand{\tablestart}{
	\interval
	\begin{longtable}{p{2cm}p{10cm}}
	\hline}
\newcommand{\tableend}{
	\hline
	\end{longtable}
	\interval}

% 改变段间隔
\setlength{\parskip}{0.2em}
\linespread{1.1}

\usepackage{lastpage}
\usepackage{fancyhdr}
\pagestyle{fancy}
\lhead{\space \qquad \space}
\chead{gawk编程 \qquad}
\rhead{\qquad\thepage/\pageref{LastPage}}
\begin{document}

\tableofcontents

\clearpage

\section{gawk程序的介绍}
\subsection{gawk程序的基本格式}
	gawk编辑器是linux环境下常用的命令行编辑器,和sed编辑器一样是流编辑器。
	gawk程序的基本格式如下所示:
	\begin{lstlisting}
	gawk [options] <program> <file>
	\end{lstlisting}

	gawk程序可用的选项如下所示:
	\begin{longtable}{p{4cm}p{6cm}}
	\hline
	-F fs & 指定描述一行中数据字段分隔符 \\
	\hline
	-f file & 指定读取程序的文件名 \\
	\hline
	-v var=value & 定义gawk程序中使用的变量和默认值 \\
	\hline
	-mf N & 指定数据文件中要处理的字段的最大数目 \\
	\hline
	-mr N & 指定数据文件中的最大记录大小 \\
	\hline
	-W keyword & 指定gawk的兼容模式或警告级别 \\
	\hline
	\end{longtable}

	gawk程序脚本由左大括号和右大括号定义,因为gawk命令行假定脚本是单文本字符串,所以必须将脚本包括在单引号内。例子如下:
	\begin{lstlisting}
	gawk '{print "Hello John!"}' file
	\end{lstlisting}

\subsection{gawk程序的数据字段变量}
	gawk可以将每行中的每个数据元素分配给数据字段变量,其中\$表示整行文本,\$1表示文本行中的第一个数据字段,
	\$2表示文本行中的第二个数据字段,\$n表示文本行中的第n个数据字段。\par
	gawk读取一行文本时,使用定义的字段分隔符描述各数据字段。gawk默认的字段分隔符为任意空白字符。\par
	使用数据字段变量的例子如下:
	\begin{lstlisting}
	gawk '{print $1}' data
	\end{lstlisting}

	可以使用“-F fs”参数项来制定字段分隔符,如下所示:
	\begin{lstlisting}
	gawk -F : '{print $1}' /etc/passwd
	\end{lstlisting}

\subsection{在程序脚本中使用多个命令}
	在程序脚本中可以使用多条命令,这些命令用分号“;”隔开,使用例子如下所示:
	\begin{lstlisting}
	echo "My name is Rich" | gawk '{$4="Dave"; print $0}'
	\end{lstlisting}

\subsection{在处理数据之前运行脚本}
	通过BEGIN关键字,可以让gawk在读取数据之前运行脚本,使用例子如下所示:
	\begin{lstlisting}
	gawk 'BEGIN {print "Hello World!"}' file
	\end{lstlisting}

	如果既想要在读取数据之前运行脚本,又想要在读取数据时候运行脚本处理数据,可以使用两个大括号,使用例子如下所示:
	\begin{lstlisting}
	gawk 'BEGIN {print "Hello World!"} {print $0}' file
	\end{lstlisting}

\subsection{在处理数据之后运行脚本}
	通过END关键字,可以让gawk在读取数据和处理数据结束之后运行脚本,使用例子如下所示:
	\begin{lstlisting}
	gawk 'BEGIN {print "Hello World!"} {print $0} END{print "byebye"}'
	\end{lstlisting}

	可以将这些技术写成一个脚本文件,例子如下:
	\begin{lstlisting}
	BEGIN
	{
		print "The latest list of users and shells"
		print " Userid		Shell"
		print " ------		------"
	}

	{
		print $1 "		" $7
	}

	END
	{
		print "This concludes the listing"
	}
	\end{lstlisting}

\section{使用变量}
\subsection{内置变量}
\subsubsection{字段和记录分隔变量}
	gawk的字段和记录分隔变量如下:
	\begin{longtable}{p{3cm}p{6cm}}
	\hline
	FILEDWIDTHS & 每个数据字段的宽度 \\
	\hline
	FS & 输入字段分隔符号 \\
	\hline
	RS & 输入记录分隔符号 \\
	\hline
	OFS & 输出字段分隔符号 \\
	\hline
	ORS & 输出记录分隔符号 \\
	\hline
	\end{longtable}

	OFS的使用例子如下所示:
	\begin{lstlisting}
	gawk 'BEGIN{FS=","; OFS="-"} {print $1,,$2,$3}' data1
	\end{lstlisting}

	FILEWIDTHS的使用例子如下所示:
	\begin{lstlisting}
	# 此方法不支持长度为变量的数据字段
	gawk 'BEGIN{FILEDWIDTHS="3 5 2 5"}{print $1,$2,$3,$4}' data1
	\end{lstlisting}

	默认情况下RS的值为换行符,所以gawk一次读入一行。如果将RS设为空值,并将数据以空行分隔,那么gawk就可以一次读入一段文本。例子如下所示:
	\begin{lstlisting}
	gawk 'BEGIN{FS="\n"; RS=""}{print $1,$4}' data
	\end{lstlisting}

\subsubsection{数据变量}
	gawk的数据变量如下所示:
	\begin{longtable}{p{3cm}p{6cm}}
	\hline
	ARGC & 出现的命令行参数的个数 \\
	\hline
	ARGIND & 当前正在处理的文件在ARGV中的索引 \\
	\hline
	ARGV & 命令行参数数组 \\
	\hline
	CONVFMT & 数字的转换格式 \\
	\hline
	ENVIRON & 当前shell环境变量及其值的关联数组 \\
	\hline
	ERRNO & 当读取或关闭输入文件时发生错误时的系统错误 \\
	\hline
	FILENAME & 用于输入到gawk程序的数据文件的文件名 \\
	\hline
	FNR & 数据文件的当前记录号 \\
	\hline
	IRNORECASE & 如果设置为非0,则忽略大小写 \\
	\hline
	NF & 数据文件中数据字段的个数 \\
	\hline
	NR & 已处理的输入记录的个数 \\
	\hline
	OFMT & 显示数字的输出格式 \\
	\hline
	RLENGTH & 匹配函数中匹配上的子字符串的长度 \\
	\hline
	RSTART & 匹配函数中匹配上的子字符串的开始索引 \\
	\hline
	\end{longtable}

\subsection{用户定义的变量}
	可以在脚本中定义变量,如下例所示:
	\begin{lstlisting}
	gawk '
	BEGIN{
	testing="This is a test"
	print testing
	testing=45
	print testing
	}'
	\end{lstlisting}

	也可以在命令行中赋值变量,如下例所示:
	\begin{lstlisting}
	# script1的内容
	BEGIN{FS=","}
	{print $n}
	# 命令行的内容
	gawk -f script1 n=2 data1
	\end{lstlisting}

	需要注意的是,在设置变量时,值不能在代码的BEGIN部分使用。
	如果要在代码的BEGIN部分之前设置变量,需要带上“-v”参数项,如下所示:
	\begin{lstlisting}
	# script2的内容
	BEGIN{print "The starting value is", n; FS=","}
	{print $n}
	# 命令行的内容
	gawk -v n=3 -f script2 data1
	\end{lstlisting}

\section{使用数组}
\subsection{定义数组变量}
	数组变量的赋值格式如下:
	\begin{lstlisting}
	var[index] = element
	\end{lstlisting}

	gawk程序的数组变量和python中的字典类似,如下例所示:
	\begin{lstlisting}
	gawk 'BEGIN{
	capital["Illinois"] = "Springfield"
	print capital["Illinois"]
	}'
	\end{lstlisting}

\subsection{在数组变量中遍历}
	如果要在gawk中遍历数组,可以使用for语句,格式如下:
	\begin{lstlisting}
	for(var in array)
	{
		statements
	}
	\end{lstlisting}

	for语句在各语句中循环,每次向变量var分配array关联数组中的下一个索引值,如下例所示:
	\begin{lstlisting}
	gawk 'BEGIN{
	var["a"] = 1
	var["g"] = 2
	var["m"] = 3
	var["u"] = 4
	for (test in var)
	{
		print "Index:", test, " - Value:", var[test]
	}
	}'
	\end{lstlisting}

	需要注意的是,索引值不是以一定的顺序返回的。

\subsection{删除数组变量}
	从关联数组删除数组索引需要一个特殊的命令,如下所示:
	\begin{lstlisting}
	delete array[index]
	\end{lstlisting}

	具体操作如下例所示:
	\begin{lstlisting}
	gawk 'BEGIN{
	var["a"] = 1
	var["g"] = 2
	for(test in var)
	{
		print "Index:",test," - Value:",var[test]
	}
	delete var["g"]
	print "------"
	for(test in var)
	{
		print "Index:",test," - Value:",var[test]
	}
	}'
	\end{lstlisting}

\section{使用模式}
\subsection{正则表达式}
	在使用正则表达式时,正则表达式必须出现在程序脚本的左括号之前,如下例所示:
	\begin{lstlisting}
	gawk 'BEGIN{FS=","} /11/{print $1}' data1
	\end{lstlisting}

	如果想让正则表达式匹配特定的数据字段,需要使用匹配操作符。
	匹配操作符需要制定数据字段变量和正则表达式,格式如下所示:
	\begin{lstlisting}
	$n ~ /^data/
	\end{lstlisting}

	使用例子如下所示:
	\begin{lstlisting}
	gawk 'BEGIN{FS=","} $2 ~ /^data2/{print $0}' data1
	\end{lstlisting}

	还可以通过!符号来否定正则表达式的匹配,从而将命令作用于没有匹配到的数据,如下例所示:
	\begin{lstlisting}
	gawk 'BEGIN{FS=","} $2 ! ~ /^data2/{print $1}' data1
	\end{lstlisting}

\subsection{数学表达式}
	gawk程序中的数学表达式如下所示:
	\begin{longtable}{p{2cm}p{5cm}}
	\hline
	x==y & 值x等于y \\
	\hline
	x<=y & 值x小等于y \\
	\hline
	x<y & 值x小于y \\
	\hline
	x>=y & 值x大等于y \\
	\hline
	x>y & 值x大于y \\
	\hline
	\end{longtable}

	具体操作如下例所示:
	\begin{lstlisting}
	gawk -F : '$1 == "root"{print $2}' /etc/passwd
	\end{lstlisting}

\section{结构化命令}
\subsection{if语句}
	gawk的if语句格式如下:
	\begin{lstlisting}
	if(condition)
	{
		statements
	}
	\end{lstlisting}

	如果只有一条命令,可以不要大括号,如下所示:
	\begin{lstlisting}
	# 格式1
	if(condition)
		statement1
	# 格式2
	if(condition) statement1
	\end{lstlisting}

	gawk的if语句还支持else分句,格式如下:
	\begin{lstlisting}
	if(condition)
	{
		statements
	}
	else
	{
		other statements
	}
	\end{lstlisting}

	还可以在单个行上使用else分句,格式如下所示:
	\begin{lstlisting}
	if(condition) statement1; else statement2
	\end{lstlisting}

\subsection{while语句}
	while语句的格式如下所示:
	\begin{lstlisting}
	while(condition)
	{
		statements
	}
	\end{lstlisting}

\subsection{do-while语句}
	do-while语句的格式如下所示:
	\begin{lstlisting}
	do
	{
		statements
	}while(condition)
	\end{lstlisting}

\subsection{for语句}
	gawk支持C语言形式的for循环,格式如下:
	\begin{lstlisting}
	for(variable assignment; condition; iteration process)
	{
		statements
	}
	\end{lstlisting}

\section{格式化打印}
	gawk的格式化打印命令是printf命令,与C语言的用法一样,格式如下:
	\begin{lstlisting}
	print "format string", var1, var2
	\end{lstlisting}

	格式化说明符的格式如下:
	\begin{lstlisting}
	%[modifier]control-letter
	\end{lstlisting}

	modifier是可选的格式化功能,如下所示:
	\begin{longtable}{p{2cm}p{5cm}}
	\hline
	width & 指定输出字段最小宽度的数字值,格式为“n” \\
	\hline
	prec & 指定浮点数中小数点右侧位数的数值,格式为“.n” \\
	\hline
	- & 将数据左对齐 \\
	\hline
	\end{longtable}

	control-letter是格式化说明符中使用的控制字,如下所示:
	\begin{longtable}{p{2cm}p{5cm}}
	\hline
	c & 将数据显示为ASCII字符 \\
	\hline
	d & 显示整数值 \\
	\hline
	i & 显示整数值 \\
	\hline
	e & 用科学计数法显示数字 \\
	\hline
	f & 显示浮点数值 \\
	\hline
	g & 以科学计数法和浮点数的较短者显示数字 \\
	\hline
	o & 显示八进制数值 \\
	\hline
	s & 显示文本字符串 \\
	\hline
	x & 显示十六进制数值,对a到f使用小写字母 \\
	\hline
	X & 显示十六进制数值,对A到F使用大写字母 \\
	\hline
	\end{longtable}

\section{内置函数}
\subsection{数学函数}
	gawk内置的数学函数如下图所示:
	\fic{1.png}

\subsection{字符串函数}
	gawk内置的字符串函数如下图所示:
	\fic{3.png}
	\fic{2.png}

\subsection{时间函数}
	gawk内置的时间函数如下图所示:
	\fic{4.png}

\section{用户定义的函数}
\subsection{定义函数}
	使用function关键字定义自己的函数,格式如下:
	\begin{lstlisting}
	function name([variable])
	{
		statements
	}
	\end{lstlisting}

	gawk的函数与C语言的函数类似,可以直接传递参数,还可以用return语句返回值。
	具体操作的例子如下所示:
	\begin{lstlisting}
	gawk '
	function myprint()
	{
		print "%-16s - %s\n", $1, $4
	}

	BEGIN
	{
		FS="\n"
		RS=""
	}

	{
		myprint()
	}' data1
	\end{lstlisting}

\subsection{创建函数库}
	gawk可以创建函数库,然后用“-f”参数项引用函数库文件。具体操作如下例所示:
	\begin{lstlisting}
	# 函数库文件名是funclib
	function myprint()
	{
		print "%-16s - %s\n", $1, $4
	}

	function myrand(limit)
	{
		return int(limit * rand())
	}

	function printthird()
	{
		print $3
	}

	# gawk脚本文件民是script4
	BEGIN
	{
		FS="\n"
		RS=""
	}

	{
		myprint()
	}
	\end{lstlisting}

	然后在命令行上用“-f”参数项制定库文件和程序文件,如下所示:
	\begin{lstlisting}
	gawk -f funclib -f script4 data1
	\end{lstlisting}

	需要注意的是,使用“-f”命令行参数时不能使用内嵌的gawk脚本,但是可以使用多个“-f”参数。

\end{document}
